\subsection*{Introducción}

{\ttfamily netcp} es la implementación de un programa cliente-\/servidor utilizado para la transferencia de ficheros múltiple y simultánea.

Hace uso de\+:
\begin{DoxyItemize}
\item Sockets en linux
\item Hilos en C++ ({\ttfamily std\+::thread})
\end{DoxyItemize}

\subsection*{El servidor}

El servidor (implementado por {\ttfamily class server}) funciona mediante mensajes (representados por {\ttfamily struct \hyperlink{structMessage}{Message}}) cuyos campos indican el estado de la transferencia. \subsubsection*{\hyperlink{structMessage}{Message}}


\begin{DoxyCode}
struct Message \{
  int code\_;
  std::array<char,1024> text;

\}
\end{DoxyCode}
 \paragraph*{codes}

5 = Primer y único mensaje

0 = Mensaje inicial 1 = Fragmento intermedio 2 = Mensaje final

3 = Solicitud para empezar envío de fichero (cli -\/$>$ sv) 4 = Confirmación para empezar envío de fichero (sv -\/$>$ cli)

\subsection*{El cliente}

{\ttfamily class client} \subsubsection*{Pasos para envío de un fichero}


\begin{DoxyEnumerate}
\item Abre el fichero y comprueba
\item Envía un \hyperlink{structMessage}{Message} con código 3 y el nombre del fichero al socket principal del server
\item Espera n segundos por la respuesta del servidor
\item 
\begin{DoxyItemize}
\item Si el servidor confirma el envío entonces empieza a enviar al nuevo puerto indicado por el servidor
\item Si no lo recibe (o recibe un código de error) lo comunica
\end{DoxyItemize}
\end{DoxyEnumerate}

\subsubsection*{Por hacer}


\begin{DoxyItemize}
\item 20/09/2020 21\+:00 Quiero crear una clase server e introducir en ella las funciones que realiza el programa en el modo server (server\+\_\+mode = true) (en proceso)
\item Sistema para el envío de ficheros. ¿\+Como diferencio entre enviar mensajes individuales y ficheros? -\/$>$ Enviar ficheros fragmentándolos en mensajes -\/$>$ Requiere establecer una estructura de mensaje que tenga campos con un código que indique si es el primero, intermedio, etc...
\item Excepciones (mirar lo hecho en el repositorio \href{https://github.com/miguel-martinr/Data-Structure-Classes}{\tt D\+SC}
\item ¿\+Podría transferir el método bind a un nivel superior en la jerarquía de Socket? -\/$>$ ¿\+Cuál es la diferencia al bindear cada tipo de socket?
\item Tengo que idear un mecanismo de comunicación entre cliente y servidor -\/$>$ El cliente solicita subir un fichero, el servidor verifica que se pueda subir (o lo pone en una cola hasta que se pueda?) y si es así abre un hilo para recibir el fichero y un socket en ese puerto para recibirlo, le comunica el puerto al cliente para que este empieze a transmitir
\end{DoxyItemize}

\subsubsection*{Hecho}


\begin{DoxyItemize}
\item 20/09/2020 21\+:11 Voy a reorganizar todo el proyecto
\begin{DoxyItemize}
\item Eliminar Socket.\+h hecho (20/09/2020 22\+:55)
\item Sacar ficheros de include/hierarchy y src/hierarchy y eliminar esos directorios (hecho 20/09/2020 22\+:55)
\item Corregir makefile para que funcione con la nueva organización (hecho 20/09/2020 22\+:55)
\end{DoxyItemize}
\end{DoxyItemize}

\subsection*{Jerarquía de clases Socket}

\begin{quote}
Esto puede considerarse un {\bfseries subproyecto} de {\ttfamily netcp} y planeo extraerlo, junto a su documentación, a un repositorio propio \+:)~\newline
~\newline
 \end{quote}
Mi idea actualmente (a fecha 20 de Septiembre de 2020) es crear una jerarquía de clases Socket que corresponda (en principio) a la siguiente estructura \+:


\begin{DoxyItemize}
\item \hyperlink{classSocket__base}{Socket\+\_\+base} \+: Clase abstracta base
\begin{DoxyItemize}
\item \hyperlink{classSocket__af}{Socket\+\_\+af} \+: Clase abstracta(?) base para aquellos sockets de la familia A\+F\+\_\+\+I\+N\+ET (protocolo I\+Pv4), tengo que corregir el nombre, el ideal sería {\ttfamily Socket\+\_\+inet} y afectaría a todos sus hijos.
\begin{DoxyItemize}
\item \hyperlink{classSocket__af__dgram}{Socket\+\_\+af\+\_\+dgram} \+: Implementación de un socket de la familia A\+F\+\_\+\+I\+N\+ET no orientado a conexión (U\+DP).
\item Socket\+\_\+af\+\_\+stream \+: Implementación de un socket de la familia A\+F\+\_\+\+I\+N\+ET orientado a conexión (T\+CP)
\end{DoxyItemize}
\item Socket\+\_\+unix \+: Clase abstracta(?) base para aquellos sockets de la familia A\+F\+\_\+\+U\+N\+IX (conexión local)
\begin{DoxyItemize}
\item Socket\+\_\+unix\+\_\+dgram \+: Implementación de un socket de la familia A\+F\+\_\+\+U\+N\+IX no orientado a conexión (U\+DP)
\item Socket\+\_\+unix\+\_\+stream \+: Implementación de un socket de la familia A\+F\+\_\+\+U\+N\+IX orientado a conexión (T\+CP) 
\end{DoxyItemize}
\end{DoxyItemize}
\end{DoxyItemize}